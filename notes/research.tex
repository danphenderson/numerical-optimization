    \documentclass[11pt]{article}

    % Packages
\usepackage{algorithm}
\usepackage{algpseudocode}
\usepackage{multicol}
\usepackage{amsmath, amssymb, amsthm, mathtools, bm}
\usepackage{geometry}
\usepackage{lipsum} % For filler text
\usepackage{array} % For better table formatting
\usepackage{float} % For forcing table placement
\usepackage{cite} % For citation
\usepackage{setspace} % For adjusting line spacing
\usepackage{listings}
\usepackage{hyperref}
\usepackage{xcolor}
\usepackage[utf8]{inputenc}
\usepackage{caption}
\usepackage{witharrows}
\usepackage{graphicx}

\captionsetup[lstlisting]{font={small}, labelfont={bf}, labelsep=colon}
\renewcommand{\lstlistingname}{Code}


% Page setup
\geometry{letterpaper, margin=.75in}

\setlength{\parindent}{0pt}
\linespread{1.25}
\renewcommand{\arraystretch}{1.25} % Adjust as needed

% Define solution environment and commonly used commands
\newenvironment{solution}
  {\renewcommand\qedsymbol{$\blacksquare$}\begin{proof}[\textbf{Solution}]}
  {\end{proof}}
\newtheorem{lemma}{Lemma}
  \renewcommand{\thelemma}{\Alph{lemma}}
\newtheorem*{lem}{Lemma}
\newtheorem{corollary}[lemma]{Corollary}
\theoremstyle{definition}
\newtheorem*{theorem}{theorem}
\theoremstyle{definition}
\newtheorem{definition}{Definition}
\theoremstyle{remark}
\newtheorem*{remark}{Remark}


\newcommand{\Z}{\mathbb{Z}}
\newcommand{\Q}{\mathbb{Q}}
\newcommand{\C}{\mathbb{C}}
\newcommand{\R}{\mathbb{R}}
\newcommand{\N}{\mathbb{N}}
\newcommand{\e}{\epsilon}
\newcommand{\p}{\rho}
\newcommand{\eps}{\epsilon}
\newcommand{\la}{\lambda}
\newcommand{\ol}{\overline}

% Custom commands for linear algebra
\newcommand{\matr}[1]{\begin{pmatrix}#1\end{pmatrix}} % For matrices
\newcommand{\vect}[1]{\begin{bmatrix}#1\end{bmatrix}} % For column vectors
\newcommand{\vct}[1]{\bm{#1}} % For bold vectors
\newcommand{\transp}{^{\mathsf{T}}} % Transpose symbol
\DeclarePairedDelimiter\ip{\langle}{\rangle} % Inner product
\DeclarePairedDelimiter\norm{\lVert}{\rVert} % Norm
\newcommand{\zero}{\mathbf{0}} % Zero vector/matrix
\newcommand{\ident}{\mathbf{I}} % Identity matrix
\DeclareMathOperator{\tr}{tr} % Trace
\DeclareMathOperator{\detm}{det} % Determinant
\DeclareMathOperator{\rank}{rank} % Rank
\DeclareMathOperator{\nullity}{nullity} % Nullity
\newcommand{\blockmat}[2]{\begin{pmatrix} #1 & #2 \end{pmatrix}} % Block matrix with 2 columns
\newcommand{\augmatr}[2]{\left(\begin{array}{#1|#1}#2\end{array}\right)}
\newcommand{\bmfour}[4]{\left( \begin{array}{c|c} #1 & #2 \\ \hline #3 & #4 \end{array} \right)} % 2x2 block matrix with lines
\newcommand{\blockvect}[2]{\begin{bmatrix} #1 \\ #2 \end{bmatrix}} % Block vector with 2 rows
% Define custom colors
\definecolor{bg}{rgb}{0.95,0.95,0.92}
\definecolor{codeblue}{rgb}{0.25,0.5,0.75}
\definecolor{keyword}{rgb}{0.5,0,0.5}
\definecolor{string}{rgb}{0,0.5,0}
% Define the style for IPython
\lstdefinelanguage{ipython}{
  basicstyle=\ttfamily\footnotesize\color{black},
  keywordstyle=\color{keyword}\bfseries,
  stringstyle=\color{string},
  backgroundcolor=\color{bg},
  morekeywords={In,Out},
  frame=lines,
  numbers=left,
  numbersep=5pt
}
% Define the style for Julia
\lstdefinelanguage{julia}{
  keywords={struct, abstract, primitive, typealias, bitstype, mutable, module, baremodule, import, importall, using, export, if, else, elseif, for, while, break, continue, function, return, type, global, local, const, let, do, try, catch, finally, end, quote, macro, where},
  basicstyle=\ttfamily\footnotesize\color{black},
  keywordstyle=\color{keyword}\bfseries,
  stringstyle=\color{string},
  backgroundcolor=\color{bg},
  frame=lines,
  numbers=left,
  numbersep=8pt,
  inputencoding=utf8,
  extendedchars=true,
  escapeinside={(*}{*)},
  literate={~}{{$\sim$}}1
           {₁}{{\textsubscript{1}}}1
           {₂}{{\textsubscript{2}}}1
           {₃}{{\textsubscript{3}}}1
           {ᵀ}{{$^{\mathrm{T}}$}}1
           {⋅}{{$\cdot$}}1
           {∈}{{$\in$}}1
           {ℝ}{{$\mathbb{R}$}}1
           {→}{{$\to$}}1
           {π}{{$\pi$}}1
           {≤}{{$\leq$}}1
           {≥}{{$\geq$}}1
           {≈}{{$\approx$}}1
           {ϵ}{{$\epsilon$}}1
           {κ}{{$\kappa$}}1
           {Δ}{{$\Delta$}}1
           {λ}{{$\lambda$}}1
}

% Define the style for Mathematica
\lstdefinelanguage{Mathematica}{
  keywords={Module,With,Block,If,Then,Else,Which,Switch,For,While,Return,Do,Table,Plot,Map,Apply,Function,Set,SetDelayed,Clear,Quit,NonlinearModelFit,FindMinimum,FindMaximum,FindRoot,NIntegrate,Integrate,Simplify,FullSimplify,DSolve,RSolve,NSolve,NDSolve,Limit,Series,Assuming,Expand,Factor,TableForm,MatrixForm,Part,Length,Dimensions,Transpose,MapThread,MapAt,Flatten,Thread,Join,Outer,ConstantArray,Riffle,ArrayPlot,Plot3D,ContourPlot,ParametricPlot,MatrixPlot,Graphics,Graphics3D,Manipulate,Evaluate,Sin,Cos,Exp,Log,Sqrt},
  keywordstyle=\color{keyword}\bfseries,
  sensitive=true,
  morecomment=[l]{(*}, morecomment=[r]{*)},
  commentstyle=\itshape\color{gray},
  morestring=[b]",
  stringstyle=\color{string},
  basicstyle=\ttfamily\footnotesize\color{black},
  backgroundcolor=\color{bg},
  frame=lines,
  numbers=left,
  numbersep=5pt,
  showstringspaces=false,
  escapeinside={(*}{*)}
}

% Optionally define a style for Mathematica code cells
\lstdefinestyle{mathematicaStyle}{
  language=Mathematica,
  frame=lines,
  backgroundcolor=\color{bg},
  basicstyle=\ttfamily\footnotesize,
  keywordstyle=\color{keyword}\bfseries,
  commentstyle=\itshape\color{gray},
  stringstyle=\color{string},
  numbers=left,
  numbersep=5pt,
  showstringspaces=false
}

    % Page header
    \title{Running Notes}
    \author{Daniel Henderson}
    \date{\today}

    % Document Start
    \begin{document}
    \maketitle

    % ------------------------------------------------------------------------------
    % SECTION 1: PRELIMINARIES -----------------------------------------------------
    % ------------------------------------------------------------------------------
    \section{Preliminaries}

    \medskip

    \subsection{Notation}
    We use the following notation\

    \textcolor{Black}{
        \begin{itemize}
            \item $\| \cdot \|$ denotes the usual $\ell_2$ norm for vectors $\vct{x}$ in $\mathbb{R}^n$ and $p = 2$ norm
            for matrices in $\mathbb{R}^{n \times m}$. i.e.,
            \begin{flalign*}
                \|x\| & := \left(\sum_i x_i^2\right)^{1/2} \\
                 \|A\| & := (\lambda_{\max}(A^\top A))^{1/2}
            = \max(\sigma(A))
            \end{flalign*}
            \item $\sigma(A) := \{\text{singular values of } A\}$.
            \item $A \in \mathbb{R}^{n\times n}$ $\implies ~ \sigma(A) = \{\text{eigenvalues of } A \text{ (i.e. spectrum)}\}$
            \item $\sigma_{\max}(A) := \max(\sigma(A))$ and $\sigma_{\min}(A) := \min(\sigma(A))$.
            \item $ A \in \mathbb{R}^{n\times n}$  $\implies \lambda_{\max}(A) := \sigma_{\max}(A)$ and $\lambda_{\min}(A) = \sigma_{\min}(A) $
            \item $\langle \cdot, \cdot \rangle$ denotes the usual inner product on $\mathbb{R}^n$, i.e.,
            $$
                \langle \vct{x}, \vct{y} \rangle := \vct{x}^\top \vct{y} = \sum_{i=1}^n \vct{x}_i \vct{y}_i = \| \vct{x} \| \|\vct{y}\| \cos(\theta)
            $$
            where $\theta$ is the angle between $\vct{x}$ and $\vct{y}$.
            \item $\mathcal{B}_r(\vct{x}) := \{ \vct{y} \in \mathbb{R}^n : \|\vct{y} - \vct{x}\| < r \}$ is the open ball of radius $r$ centered at $\vct{x} \in \R^n$.
        \end{itemize}
    }

    \medskip

    \subsection{Assumptions}
    We assume the following assumptions\

    \textcolor{Black}{
        \begin{itemize}
            \item Let $\Omega \subset \R^n$ be a bounded open subset with a smooth boundary $\partial \Omega$,
            which we assume is a $C^2$-manifold. Additionally, we often consider $\Omega$ and $\partial \Omega$
            to be the $n-1$-dimensional hypersurface of a $n$-dimensional ball.
            \item Let be a function $f \in C^\infty(\Omega; \R)$ is a continous
            and infinitely-differentiable mapping from $\Omega$ to $\R$.
            In general, $f$ is nonlinear and nonconvex on $\Omega$.
            \item We assume $C^2$ regularity of the boundary $\partial \Omega$
            ao that the derivatives of $f$ may be extended to $\partial \Omega$.
            \item The typefont $\vct{x}$ will denote a vector in $\mathbb{R}^n$.
        \end{itemize}
    }

    \subsection{Definitions}
    Using the above notation, we assume our assumptions hold, and we define the following\
    % Lipschitz
    \begin{definition}
    \textcolor{blue}{
        $f$ is \textbf{$L$-Lipschitz} if $\forall ~ \vct{x_1},\vct{x_2}$
        $$
        \; \exists~ L \geq 0 ~:~  \| f(\vct{x_1}) - f(\vct{x_2})\| 
        \leq L\|\vct{x_1} - \vct{x_2}\|
        $$
    }
    \end{definition}

    % L-Lipschitz gradient
    \begin{definition} 
    \textcolor{blue}{
        $f$ has \textbf{$\ell$-Lipschitz gradient}, or, $f$ is \textbf{$\ell$-smooth} if
        $\forall ~ \vct{x_1},\vct{x_2}$
        $$
            \exists~ \ell \geq 0 ~:~ \|\nabla f(\vct{x_1}) - \nabla f(\vct{x_2})\| 
            \leq \ell\|\vct{x_1} - \vct{x_2}\|
        $$
    }
    \end{definition}

    % Rho Lipschitz Hessian
    \begin{definition}
    \textcolor{blue}{
        $f$ has \textbf{$\rho$-Lipschitz Hessian} if $\forall ~ \vct{x_1},\vct{x_2}$
        $$
            \; \exists~ \rho \geq 0 ~:~ \|\nabla^2 f(\vct{x_1}) - \nabla^2 f(\vct{x_2})\| 
            \leq \rho\|\vct{x_1} - \vct{x_2}\|
        $$
    }
    \end{definition}

    % Convexity
    \begin{definition}
    \textcolor{blue}{
        $f$ is \textbf{convex} if $\forall ~ \vct{x_1},\vct{x_2}$
        \begin{flalign*}
            f(\vct{x_2}) & \geq f(\vct{x_1}) + \langle  \vct{x_2} - \vct{x_1}, \nabla f(\vct{x_1}) \rangle\\
                        & = f(\vct{x_1}) + \nabla f(\vct{x_1})^T(\vct{x_2} - \vct{x_1}) 
        \end{flalign*}
    }
    \end{definition}

    % Strictly Convex
    \begin{definition}
    \textcolor{blue}{
        $f$ is \textbf{strictly convex} if
        \begin{flalign*}
            \exists~ \mu > 0 ~ : ~ & \nabla^2 f \succeq \mu I \\ 
            & \iff \lambda_{\min}(\nabla^2 f) \geq \mu > 0 
        \end{flalign*}
    }
    \end{definition}

    % Alpha strong convexity
    \begin{definition}
    \textcolor{blue}{
        $f$ is \textbf{$\alpha$-strongly convex} if $\forall ~ \vct{x_1},\vct{x_2} \exists~ \alpha > 0$ s.t.
        \begin{flalign*}
            & f(\vct{x_2}) \geq f(\vct{x_1}) + \langle \nabla f(\vct{x_1}), \vct{x_2} - \vct{x_1} \rangle + 
            \frac{\alpha}{2}\|\vct{x_2} - \vct{x_1} \|^2\\
            & \iff \lambda_{\min}(\nabla^2 f (\vct{x})) \geq - \alpha. 
        \end{flalign*}
    }
    \end{definition}

    % First-order stationary point
    \begin{definition}
    \textcolor{blue}{
        $\vct{x^*}$ is a \textbf{first-order stationary point}
        if $\| \nabla f(x^*) \| = 0$.
    }
    \end{definition}

    % Eps-first-order stationary point
    \begin{definition}
    \textcolor{blue}{
        $\vct{x^*}$ is an \textbf{$\eps$-first-order stationary point} 
        if $\|\nabla f(x^*)\| \leq \eps$. 
    }
    \end{definition}

    % Second-order stationary point
    \begin{definition}
    \textcolor{blue}{
        $\vct{x}^* \in \R^n$ is a \textbf{second-order stationary point} if 
        $
            \| \nabla f(x^*) \| = 0\text{ and } \nabla^2 f(x^*) \succeq 0.
        $
    }
    \end{definition}

    % Eps-second-order stationary point
    \begin{definition}
    \textcolor{blue}{
        if $f$ has $\rho$-Lipschitz Hessian, $\vct{x}^* \in \R^n$ is a
        \textbf{$\eps$-second-order stationary point} if 
        $$
            \| \nabla f(x^*) \| \leq \eps \text{ and }\nabla^2 f(x^*) \succeq -\sqrt{\rho \eps}
        $$
        \begin{remark}
        Note that the Hessian is not required to be positive definite, 
        but it is required to have a small eigenvalue. 
    \end{remark}
    }
    \end{definition}


    % General Form of Unconstrained Optimization Problem
    \begin{definition}
    \textcolor{blue}{
    We consider the general form of our \textbf{unconstrained optimization problem} to be
    \begin{flalign}
        \min_{\vct{x} \in \mathbb{R}^n} f(\vct{x}) \quad \text{s.t.} \quad 
        \vct{x} \in K \subseteq \mathbb{R}^n
        \label{eq:unconstrained-opt}
    \end{flalign}
    where $K$ is a compact set in $\mathbb{R}^n$.
    We denote the \textbf{optimal solution} and \textbf{optimal value} 
    of the optimization problem 
    \begin{flalign*}
        \vct{x}^* & = \mathop{\mathrm{arg\,min}}_{\vct{x} \in K} f(\vct{x})\\
        \vct{f}^* & = \mathop{\mathrm{min}}_{\vct{x} \in K} f(\vct{x})
    \end{flalign*}
    where $\vct{x}^*$ satisfies the first-order optimality condition, i.e., $\nabla f(\vct{x}^*) = 0$.
    }
    \end{definition}

    \begin{definition}
    \textcolor{blue}{
        Let the \textbf{gradient flow} of $f$ be a solution to the dynamical system defined as
        $$
            \vct{\gamma}'(t) = -\nabla f(\vct{\gamma}(t)) 
        $$
        where the evolution of our phase space is driven by the negative gradient of $f$.
        A \textbf{gradient flow line} of $f$ is an integral curve $\vct{\gamma} : [0, t_f] \rightarrow \Omega$
        satisfying the above evolutionary system (ordinary-differential equation) subject to $\vct{\gamma}(0) = \vct{x_0}$.
    }
    \end{definition}

    \medskip

    We aim to classify the phase space of the \emph{gradient flow} of $f$ on $\Omega$.
    First we notice that for and any critical point $\vct{x}^*$, 
    \begin{flalign*}
        \vct{\gamma}(t) & = \vct{x}^* ~ \forall ~ t \in [0, t_f] \\
        \implies & \quad \vct{\gamma}'(t) = \vct{0} ~\text{ and }~
        -\nabla f(\vct{\gamma}(t)) = -\nabla f(\vct{x}^*) = \vct{0} \quad \because ~ \vct{x}^* ~ \text{ is a critical point} \\
        \therefore & \quad \vct{\gamma}(t) = -\nabla f(\vct{\gamma}(t))  ~ \forall ~ t \in [0, t_f]
    \end{flalign*}

    Consequently, by the uniqueness of solutions for ordinary differential
    equations, if any flow line contains a \emph{first-order} critical point $\vct{x}^*$,
    it must be a constant flow line.

    \begin{lemma}
        The function $f: \omega \rightarrow \R$ is nonincreasing
        along any flow-line $\vct{\gamma}(t)$ and strictly decreasing
        along flow lines not containing a critical point $x^*$.
    \end{lemma}
    \begin{proof}
        Let $\vct{\gamma} : [0, t_f] \rightarrow \Omega$ be a flow line.
        Consider the composition $f \circ \vct{\gamma} : [0, t_f] \rightarrow \R$,
        its derivative is
        \begin{flalign*}
            \dfrac{d}{dt}\big( f(\vct{\gamma}(t)) \big) & =
                \langle \nabla_{\vct{\gamma(t)}}(f), \dfrac{d\vct{\gamma}}{dt} \rangle \\
                 & = \langle \nabla_{\vct{\gamma(t)}}(f), - \nabla_{\gamma(t)}(f) \rangle \\
                 & = - \langle \nabla_{\vct{\gamma(t)}}(f), \nabla_{\gamma(t)}(f) \rangle \\
                 & = - | \nabla_{\vct{\gamma(t)}}(f) |^2 \\
                 & \leq 0
        \end{flalign*}
        Therefore, $f'(\vct{\gamma(t)}) = 0$ iff $\vct{\gamma}(t)$ is on a critical point
        of $f$. In particular, if $\vct{\gamma}(t)$ does not contain in its image
        a critical point of $f$, then the above inequality implies that
        $f$ is strictly decreasing along the integral curve $\vct{\gamma(t)}$.
    \end{proof}

    \medskip

    \begin{theorem}
        For all $\vct{x}$ in the closed manifold $\overline{\Omega}$, there exists
        uniquely $\vct{\gamma}_{\vct{x}}(t) : \R \rightarrow \overline{\Omega}$
        such that $\vct{\gamma}_{\vct{x}}(0) = \vct{x}$
        and the limits
        $$
            \lim_{t \to -\infty } \vct{\gamma}_{\vct{x}}(t) ~ \text{ and } ~ 
            \lim_{t \to \infty } \vct{\gamma}_{\vct{x}}(t)
        $$
        exist and converge to \emph{critical-points} of $f$.
    \end{theorem}
   
    Now it may be shown that the flow map operator $T : \overline{\Omega} \times \R \rightarrow \overline{\Omega}$
    defined as $T(\vct{x}, t) : =  \vct{\gamma}_{\vct{x}}(t)$. This implies theoretically that we may
    perform analysis of our phase space as constructed by a smooth union of integral curves.
    Consequently, by our previous lemma, if the flow map $T$ contains a critical point $\vct{x}^*$ then
    it ought to be that $T(\vct{x}^*, t) \equiv \text{ const } ~ \forall t$, otherwise,
    $T$ is descending for all of time.

   

    % Scheme for solving unconstrained optimization problem
    \begin{definition}
        A \textbf{scheme} for solving a general form \emph{unconstrained optimization problem}
        is a one-parameter family of iteration operators:
        $$
            T_h : \R^n \rightarrow \R^n ~ \text{ where } ~\vct{x}_{k+1} = T_h(\vct{x}_k), ~ h \in (0, h_0]
        $$
        where $h_0$ is a constant and $h$ is the step size. The scheme is well-defined such that
        the triplet $(\vct{x}_0, h, T_h)$ satisfy
        \begin{enumerate}
            \item \emph{Consistency}: $\forall \vct{x} \in K$ 
            $$
                \left( T_h(\vct{x}) - \vct{x} \right) h^{-1} + \nabla f(\vct{x}) \rightarrow 0 \text{ as } h \rightarrow 0.
            $$
            implying that a single step approximates the continous gradient flow w/ local error
            $\mathcal{O}(h^{p+1})$ where $p$ is the global order of the scheme.
            \item \emph{Stability}: $\exists~ c > 0, h_0 > 0 ~:~ \forall h \in (0, h_0]$,
            and for all $\vct{x_1}, \vct{x_2}$ in a \emph{neighborhood $N \subset K$}
            around an \emph{optimal solution} $\vct{x}^*$
            $$
                \| T_h(\vct{x_1}) - T_h(\vct{x_2}) \| \leq (1 - ch) \| \vct{x_1} - \vct{x_2} \|
            $$
            where $c$ is a constant that depends on the scheme and $h$ is the step size.
            Or, equivalently, the scheme is \emph{stable} if
            $\exists~ c > 0, h_0 > 0 ~:~ \forall h \in (0, h_0]$,
            and for all $\vct{x}$ in a \emph{neighborhood} about $\vct{x}^*$,
            each step results in a strict decrease of by atleast a factor of $1 - ch$, i.e.,
            $$
            \| J(T_h(\vct{x})) \| \leq (1 - ch)
            $$
            where $J(T_h(\vct{x}))$ is the Jacobian of the scheme.
            \item \emph{Convergence}: $\forall~ \vct{x_0} \in N \subset K$, s.t. $N$ is some neighborhood
            around a strict minimizer $x^*$. and $\forall \eps > 0$ 
            $$
                \exists~ K \in \mathbb{N} ~:~ \forall k ~ > K, \vct{x}_k \in N \text{ and } 
                d(\vct{x}_k, \vct{x}^*) \leq \eps.
            $$
        \end{enumerate}
    \end{definition}

    % Strict Saddle Property
    \begin{definition}
        \textcolor{blue}{
            $\vct{x}^*$ is \textbf{non-degenerate} if $\nabla^2 f(x^*)$ is non-singular.
        }
    \end{definition}


    % Level Set
    \begin{definition}
        \textcolor{blue}{
            The \textbf{level set} of $f$ at $c$ is the set of points $\vct{x} \in \Omega$ such that
            $f(\vct{x}) = c$, i.e.,
            $$
                L_c = \{ \vct{x} \in \Omega : f(\vct{x}) = c \}
            $$
            The level set $L_c$ is a smooth manifold with boundary $\partial L_c$.
                        The \textbf{sublevel} and \textbf{superlevel} sets of $f$ at $c$
            are the sets of points $\vct{x} \in \Omega$ such that
            $f(\vct{x}) \leq c$ and $f(\vct{x}) \geq c$, respectively, i.e.,
            \begin{flalign*}
                L_c^- & = \{ \vct{x} \in \Omega : f(\vct{x}) \leq c \} \\
                L_c^+ & = \{ \vct{x} \in \Omega : f(\vct{x}) \geq c \}.
            \end{flalign*}
            The sublevel set $L_c^-$ is a smooth manifold with boundary $\partial L_c^-$
            and the superlevel set $L_c^+$ is a smooth manifold with boundary $\partial L_c^+$.
        }
    \end{definition}

    \medskip 

    \begin{theorem}
            For a Morse function $f$ on $\Omega$, the gradient of $f$ is either zero or
            orthogonal to the tangent space of the level set $L_c$ at $\vct{x} \in L_c$.
    \end{theorem}

    The above theorem implies that at a stationary point $\vct{x^*}$, a
    level set $L_{\vct{x^*}}$ is reduced to a single point
    when $\vct{x^*}$ is a local minimum or maximum. 
    Otherwise, the level set may have a singularity
    such as a self-intersection or a cusp.
    \medskip

    % SECTION 2: OVERVIEW OF OPTIMIZATION SCHEMES ----------------------------------
    % ------------------------------------------------------------------------------
    \section{Overview of Optimization Schemes}
    % ------------------------------------------------------------------------------

    % Gradient Descent 
    \subsection{Gradient Descent (GD)}
    \subsubsection*{Continous Model}
        \textcolor{Black}{
            TODO: Explain the gradient descent phase space and the level
            set equations for the continous model.
        }
    \subsubsection*{Scheme}
        The \textbf{gradient descent} line search scheme is
        $$
            T_h(\vct{x}) = \vct{x} - h \nabla f(\vct{x})
        $$
        first-order $(p = 1)$ and contractive when $\nabla^2 f \succeq \mu I \succeq 0$.

    \medskip


    % Newton's Method
    \subsection{Newton's Method (NM)}
    \subsubsection*{Continous Model}
        \textcolor{Black}{
            TODO: Explain the Newton's method phase space and the level
            set equations for the continous model.
        }

    \subsubsection*{Scheme}
        The \textbf{Newton's method} line search scheme is
        $$
            T_h(\vct{x}) = \vct{x} - h \nabla^2 f(\vct{x})^{-1} \nabla f(\vct{x})
        $$
        second-order $(p = 2)$ and contractive when $\nabla^2 f \succeq \mu I \succeq 0$.
    \medskip

    % Trust Region Methods (TR)
    \subsection{Trust Region Methods (TR)}
    \subsubsection*{Continous Model}
        \textcolor{Red}{
            TODO: Explain the trust region method phase space and the level
            set equations for the continous model.
        }
    \subsubsection*{Scheme}
        The \textbf{trust region method} line search scheme is
        $$
            T_h(\vct{x}) = \vct{x} + \arg \min_{\vct{\tau}} m_{\vct{x}}(\vct{\tau})
        $$ 
        where $m_{\vct{x}}(\vct{\tau}) = f(\vct{x}) + \langle \nabla f(\vct{x}), \vct{\tau} \rangle +
        \frac{1}{2} \langle \vct{\tau}, \nabla^2 f(\vct{x}) \vct{\tau} \rangle$ is the 
        quadratic approximation of $f$ at $\vct{x}$ and $\|\vct{\tau}\| \leq \Delta$
        is the trust region constraint.

    \medskip


    % Quasi-Newton Methods
    \subsection{Quasi-Newton Methods (QN)}
    \subsubsection*{Continous Model}
        \textcolor{Red}{
            TODO : Explain the quasi-newton method phase space and the level
            set equations for the continous model.
        }
    \subsubsection*{Scheme}
        The \textbf{quasi-newton method} line search scheme is
        $$
            T_h(\vct{x}) = \vct{x} - h  B \nabla f(\vct{x})
        $$ 
        where $B \approx \nabla^2 f^{-1}(\vct{x})$ is a positive-definite approximation of the Hessian.
        The quasi-newton method is a first-order $(p = 1)$ scheme and contractive when
        $\nabla^2 f \succeq \mu I \succeq 0$.


    \medskip

    % Section 3: Theory ------------------------------------------------------
    % ------------------------------------------------------------------------------
    \section{Theory}
    % ------------------------------------------------------------------------------


    \subsection{Morse Theory}
    \medskip

    Note that $\Omega$ is a bounded subset of $\R^n$, so its closure
    $\overline{\Omega} = \Omega \cup \partial \Omega$ is a compact subset
    of $\R^n$, by the Heine-Borel Theorem.
    Also, the boundary $\partial \Omega$ is sufficiently smooth,
    so we can apply the theory of smooth manifolds.
    The closure $\overline{\Omega}$ is a compact subset of $\R^n$ and
    is a smooth manifold with boundary $\partial \Omega$.
    The interior $\Omega$ is an open subset of $\R^n$ and is a smooth manifold.


    \begin{definition}
        \textcolor{blue}{
            A point $\vct{x^*} \in \Omega$ is a \textbf{critical point} of $f$ if the
            differentiable map $df_p: T_p \Omega \rightarrow \R$ is zero. (Here
            $T_p \Omega$ is a tangent space of the Manifold $M$ at $p$.) 
            The set of critical points of $f$ is denoted by $\text{crit}(f)$.
        }
    \end{definition}

    \begin{definition}
        A point $\vct{x^*} \in \Omega$ is a \textbf{non-degenerate critical point} of $f$ if
        the Hessian $H_p f$ is non-singular.
    \end{definition}

    \begin{definition}
        The \textbf{index} of a \emph{non-degenerate critical point} $\vct{x^*}$ is defined to be
        the dimension of the negative eigenspace of the Hessian $H_p f$.
        \begin{itemize}
            \item local minima at $\vct{x^*}$ have index $0$.
            \item local maxima at $\vct{x^*}$ have index $n$.
            \item saddle points at $\vct{x^*}$ have index $k$ where $0 < k < n$.
        \end{itemize}
        We reserve the integers $c_0, c_1, \dots, c_i, \dots, c_n$ to denote the number of
        critical points of index $i$.
    \end{definition}

    \begin{remark}
        For each objective function $f$ we are interested in determining the
        critical points of $f$ 
    \end{remark}

    \begin{remark}
        The \textbf{Morse function} is a smooth function $f : \Omega \rightarrow \R$ such that
        all critical points of $f$ are non-degenerate.
    \end{remark}

    \begin{theorem}
        \textcolor{Blue}{
            Let $f$ be a Morse function on $\Omega$, then the Euler characteristic of $\Omega$ is
            given by
            $$
                \chi(\Omega) = \sum_{i=0}^n (-1)^i c_i
            $$
            where $c_i$ is the number of critical points of index $i$.
        }
    \end{theorem}

    \begin{remark}
        The Euler characteristic $\chi(\Omega)$ is a topological invariant of the manifold $\Omega$
        and is independent of the choice of Morse function $f$.
        The Euler characteristic is a measure of the "shape" of the manifold and can be used to
        distinguish between different topological spaces.
        The Euler characteristic may be defined by the alternating sum of the Betti numbers
        $b_i$ of the manifold $\Omega$
        $$
            \chi(\Omega) = \sum_{i=0}^n (-1)^i b_i
        $$
        where $b_i$ is the $i$-th Betti number of the manifold $\Omega$.
    \end{remark}

    \begin{theorem}
        \textcolor{ForestGreen}{
            (Sard's theorem) Let $f$ be a Morse function on $\Omega$, then
            the image $f(\text{crit}(f))$ has Lebesque measure zero in $\R$.
        }
    \end{theorem}

    \begin{remark}
        We state a particular instance of Sard's theorem for continous scalar-valued functions $f$,
        which was first proved by Anothony P. Morse in 1939.
        The theorem asserts that the image of the critical points of a Morse function is a set 
        of measure zero in $\R$. This means that the critical points of a Morse function are "rare" in the sense that they
        do not form a dense subset of the manifold $\Omega$.
        Consequently, selecting $\vct{x} \in \Omega$ at random will almost never yeild a critical
        point of $f$.
    \end{remark}

    \begin{remark}
        The property that $\vct{x^*} \in \Omega$ being a \emph{critical point} of a Morse function $f$ is
        not dependent of the metric of $\Omega \subset \R^n$ (and consequently, the norm induced by the metric)
    \end{remark}



    \subsection{Analysis of Gradient Descent (GD)}

        \begin{theorem} Assume $f$ is $\ell$-smooth and $\alpha$-strongly convex and that $\eps > 0$.
            If we iterate the gradient descent \emph{scheme} with $h = h_0 = \frac{1}{\ell}$ held fixed, i.e.,
            $$
                T_{h}(\vct{x}_k) = \vct{x}_k - \frac{1}{\ell} \nabla f(\vct{x}_k),
            $$
            then $d(x_k, x^*) \leq \eps$ for all $k > K$ where $K$ is chosen to satisfy
            $$
                \frac{2\ell}{\alpha} \cdot \log\left(\frac{d(x_0, x^*)}{\eps}\right) \leq K.
            $$
        \end{theorem}

        \begin{remark}
            Under \emph{$\ell$-smoothness} and \emph{$\alpha$-strong convexity} assumptions
            in a neighborhood $\Omega$ about $\vct{x}^*$, it may be shown directly from 
            the above theorem above that the \emph{GD scheme} converges linearly to 
            the optimal solution $\vct{x}^*$ at a rate of
            $$
                \frac{d(\vct{x}_k, \vct{x}^*)}{d(\vct{x}_{k-1}, \vct{x}^*)} \leq 1 - \frac{\alpha}{\ell}
            $$
            where $d(\vct{x}_k, \vct{x}^*)$ is the distance between the current iterate $\vct{x}_k$ and 
            the optimal solution $\vct{x}^*$. The convergence rate is linear in the sense that the 
            distance between the current iterate and the optimal solution decreases by a factor of 
            $1 - \frac{\alpha}{\ell}$ at each iteration. (Ref: TODO)
        \end{remark}

        \begin{remark}
            Convergence to a first-order stationary point trivally
            implies convergence to a $\eps$-first-order stationary point.
            Similarly, convergence to a second-order stationary point trivially
            implies convergence to a $\eps$-second-order stationary point.
        \end{remark}

        \medskip

        \begin{theorem}
            Assume $f$ is $\ell$-smooth, then for any $\eps > 0$, if we iterate
            the GD scheme with $h = h_0 = \frac{1}{\ell}$ held fixed starting from
            $\vct{x}_0 \in \Omega$ where $\Omega$ is a neighborhood of $x^*$,
            then the number of iterations $K$ required to achieve the stopping condition
            $\| \nabla f(\vct{x}_k) \| \leq \eps$ is at most
            $$
                \left\lceil \frac{\ell}{\eps^{2}} \, (f(\vct{x}_0) - f(\vct{x}^*)) \right\rceil
            $$
        \end{theorem}

        \begin{remark}
            \textbf{TODO and Questions}
            \begin{itemize}
                \item State how we use theorems in when performing analysis from
                the results of our experimewts.
                \item What is the relationship between $\ell$ and $\alpha$?
                \item In practice do we know how to compute $\ell$ and $\alpha$?
                \item What is the relationship between $\ell$ and $\rho$?
                \item In practice do we know how to compute $\ell$ and $\rho$?
            \end{itemize}
        \end{remark}


    \begin{remark}
        JinSaddle.pdf Section 3.1 - Strict Saddle Property
    \end{remark}


    \end{document}
    %% END OF DOCUMENT --------------------------------------------------------------
