\documentclass[10pt]{article}

% Packages
\usepackage{algorithm}
\usepackage{algpseudocode}
\usepackage{multicol}
\usepackage{amsmath, amssymb, amsthm, mathtools, bm}
\usepackage{geometry}
\usepackage{lipsum} % For filler text
\usepackage{array} % For better table formatting
\usepackage{float} % For forcing table placement
\usepackage{cite} % For citation
\usepackage{setspace} % For adjusting line spacing
\usepackage{listings}
\usepackage{hyperref}
\usepackage{xcolor}
\usepackage[utf8]{inputenc}
\usepackage{caption}
\usepackage{witharrows}
\usepackage{graphicx}

\captionsetup[lstlisting]{font={small}, labelfont={bf}, labelsep=colon}
\renewcommand{\lstlistingname}{Code}


% Page setup
\geometry{letterpaper, margin=.75in}

\setlength{\parindent}{0pt}
\linespread{1.25}
\renewcommand{\arraystretch}{1.25} % Adjust as needed

% Define solution environment and commonly used commands
\newenvironment{solution}
  {\renewcommand\qedsymbol{$\blacksquare$}\begin{proof}[\textbf{Solution}]}
  {\end{proof}}
\newtheorem{lemma}{Lemma}
  \renewcommand{\thelemma}{\Alph{lemma}}
\newtheorem*{lem}{Lemma}
\newtheorem{corollary}[lemma]{Corollary}
\theoremstyle{definition}
\newtheorem*{theorem}{theorem}
\theoremstyle{definition}
\newtheorem{definition}{Definition}
\theoremstyle{remark}
\newtheorem*{remark}{Remark}


\newcommand{\Z}{\mathbb{Z}}
\newcommand{\Q}{\mathbb{Q}}
\newcommand{\C}{\mathbb{C}}
\newcommand{\R}{\mathbb{R}}
\newcommand{\N}{\mathbb{N}}
\newcommand{\e}{\epsilon}
\newcommand{\p}{\rho}
\newcommand{\eps}{\epsilon}
\newcommand{\la}{\lambda}
\newcommand{\ol}{\overline}

% Custom commands for linear algebra
\newcommand{\matr}[1]{\begin{pmatrix}#1\end{pmatrix}} % For matrices
\newcommand{\vect}[1]{\begin{bmatrix}#1\end{bmatrix}} % For column vectors
\newcommand{\vct}[1]{\bm{#1}} % For bold vectors
\newcommand{\transp}{^{\mathsf{T}}} % Transpose symbol
\DeclarePairedDelimiter\ip{\langle}{\rangle} % Inner product
\DeclarePairedDelimiter\norm{\lVert}{\rVert} % Norm
\newcommand{\zero}{\mathbf{0}} % Zero vector/matrix
\newcommand{\ident}{\mathbf{I}} % Identity matrix
\DeclareMathOperator{\tr}{tr} % Trace
\DeclareMathOperator{\detm}{det} % Determinant
\DeclareMathOperator{\rank}{rank} % Rank
\DeclareMathOperator{\nullity}{nullity} % Nullity
\newcommand{\blockmat}[2]{\begin{pmatrix} #1 & #2 \end{pmatrix}} % Block matrix with 2 columns
\newcommand{\augmatr}[2]{\left(\begin{array}{#1|#1}#2\end{array}\right)}
\newcommand{\bmfour}[4]{\left( \begin{array}{c|c} #1 & #2 \\ \hline #3 & #4 \end{array} \right)} % 2x2 block matrix with lines
\newcommand{\blockvect}[2]{\begin{bmatrix} #1 \\ #2 \end{bmatrix}} % Block vector with 2 rows
% Define custom colors
\definecolor{bg}{rgb}{0.95,0.95,0.92}
\definecolor{codeblue}{rgb}{0.25,0.5,0.75}
\definecolor{keyword}{rgb}{0.5,0,0.5}
\definecolor{string}{rgb}{0,0.5,0}
% Define the style for IPython
\lstdefinelanguage{ipython}{
  basicstyle=\ttfamily\footnotesize\color{black},
  keywordstyle=\color{keyword}\bfseries,
  stringstyle=\color{string},
  backgroundcolor=\color{bg},
  morekeywords={In,Out},
  frame=lines,
  numbers=left,
  numbersep=5pt
}
% Define the style for Julia
\lstdefinelanguage{julia}{
  keywords={struct, abstract, primitive, typealias, bitstype, mutable, module, baremodule, import, importall, using, export, if, else, elseif, for, while, break, continue, function, return, type, global, local, const, let, do, try, catch, finally, end, quote, macro, where},
  basicstyle=\ttfamily\footnotesize\color{black},
  keywordstyle=\color{keyword}\bfseries,
  stringstyle=\color{string},
  backgroundcolor=\color{bg},
  frame=lines,
  numbers=left,
  numbersep=8pt,
  inputencoding=utf8,
  extendedchars=true,
  escapeinside={(*}{*)},
  literate={~}{{$\sim$}}1
           {₁}{{\textsubscript{1}}}1
           {₂}{{\textsubscript{2}}}1
           {₃}{{\textsubscript{3}}}1
           {ᵀ}{{$^{\mathrm{T}}$}}1
           {⋅}{{$\cdot$}}1
           {∈}{{$\in$}}1
           {ℝ}{{$\mathbb{R}$}}1
           {→}{{$\to$}}1
           {π}{{$\pi$}}1
           {≤}{{$\leq$}}1
           {≥}{{$\geq$}}1
           {≈}{{$\approx$}}1
           {ϵ}{{$\epsilon$}}1
           {κ}{{$\kappa$}}1
           {Δ}{{$\Delta$}}1
           {λ}{{$\lambda$}}1
}

% Define the style for Mathematica
\lstdefinelanguage{Mathematica}{
  keywords={Module,With,Block,If,Then,Else,Which,Switch,For,While,Return,Do,Table,Plot,Map,Apply,Function,Set,SetDelayed,Clear,Quit,NonlinearModelFit,FindMinimum,FindMaximum,FindRoot,NIntegrate,Integrate,Simplify,FullSimplify,DSolve,RSolve,NSolve,NDSolve,Limit,Series,Assuming,Expand,Factor,TableForm,MatrixForm,Part,Length,Dimensions,Transpose,MapThread,MapAt,Flatten,Thread,Join,Outer,ConstantArray,Riffle,ArrayPlot,Plot3D,ContourPlot,ParametricPlot,MatrixPlot,Graphics,Graphics3D,Manipulate,Evaluate,Sin,Cos,Exp,Log,Sqrt},
  keywordstyle=\color{keyword}\bfseries,
  sensitive=true,
  morecomment=[l]{(*}, morecomment=[r]{*)},
  commentstyle=\itshape\color{gray},
  morestring=[b]",
  stringstyle=\color{string},
  basicstyle=\ttfamily\footnotesize\color{black},
  backgroundcolor=\color{bg},
  frame=lines,
  numbers=left,
  numbersep=5pt,
  showstringspaces=false,
  escapeinside={(*}{*)}
}

% Optionally define a style for Mathematica code cells
\lstdefinestyle{mathematicaStyle}{
  language=Mathematica,
  frame=lines,
  backgroundcolor=\color{bg},
  basicstyle=\ttfamily\footnotesize,
  keywordstyle=\color{keyword}\bfseries,
  commentstyle=\itshape\color{gray},
  stringstyle=\color{string},
  numbers=left,
  numbersep=5pt,
  showstringspaces=false
}

% Page header
\title{MA5680: TRS}
\author{Daniel Henderson}
\date{\today}


% Page body
\begin{document}
\maketitle

\noindent In the language of your choice implement step 2 of algorithm 5.1 on p14 of the article.

\begin{solution}
    A bit of a digression but we show that out generalized eigenvalue problem may be written in standard form:
    \begin{flalign*}
        \bmfour{-B}{A}{A}{-\frac{gg^\top}{\Delta^2}} \vect{y_1 \\ y_2} & 
            = -\la^* \bmfour{0}{B}{B}{0} \vect{y_1 \\ y_2}  \\
        \bmfour{0}{B}{B}{0}^{-1} \bmfour{-B}{A}{A}{-\frac{gg^\top}{\Delta^2}} \vect{y_1 \\ y_2} & = 
            -\la^* \bmfour{0}{B}{B}{0}^{-1} \bmfour{0}{B}{B}{0} \vect{y_1 \\ y_2} \\
        \bmfour{0}{B^{-1}}{B^{-1}}{0} \bmfour{-B}{A}{A}{-\frac{gg^\top}{\Delta^2}} \vect{y_1 \\ y_2} & 
            = -\la^* \bmfour{0}{B^{-1}}{B^{-1}}{0} \bmfour{0}{B}{B}{0} \vect{y_1 \\ y_2} \\
        \bmfour{-B^{-1} B}{-B^{-1} \frac{gg^\top}{\Delta^2}}{-B^{-1} A}{B^{-1} A} \vect{y_1 \\ y_2} = &
            = -\la^* \bmfour{B^{-1}B}{0}{0}{B^{-1}B} \vect{y_1 \\ y_2} \\
        \bmfour{-B^{-1} B}{-B^{-1} \frac{gg^\top}{\Delta^2}}{-B^{-1} A}{B^{-1} A} \vect{y_1 \\ y_2} = &
            -\la^*I_{2n \times 2n} \vect{y_1 \\ y_2} \\
        \left(\bmfour{-B^{-1} B}{-B^{-1} \frac{gg^\top}{\Delta^2}}{-B^{-1} A}{B^{-1} A}  - \la^* I_{2n \times 2n} \right)\vect{y_1 \\ y_2} & = 0
    \end{flalign*}
    Note, that $\bmfour{0}{B}{B}{0}$ is invertible, as $\det \bmfour{0}{B}{B}{0} = \det (0_{n \times n} - B) * \det (0_{n \times n} + B) = -\det(B)^2 < 0$ as
    $B$ is symmetric positive definite, which implies $\det(B) > 0$. Concluding our digression, we make note that 
    the hard-case occurs when $\la^*$ equals the largest eigenvalue of $A + \la B$, or equivalently, when $\text{alge}(\la^*) > 1$.
    
    \newpage
    
\lstset{language=julia}
    \begin{lstlisting}[language=julia, caption={Implementation of Algorithm 5.1 step 2}]
julia> using LinearAlgebra
julia> A = let X = randn(n, n); X + X'; end; 
julia> B = let X = randn(n, n); X * X'; end;
julia> Δ = 0.1
julia> g, p0 = randn(n), randn(n)
julia> m = (p) -> p' * g + 0.5 * p' * A * p
julia> M0 = [-B A; A -g*g'/Δ^2];
julia> M1 = -[zeros(n,n) B; B zeros(n,n)]
julia> F = eigen(M0, M1)
julia> λstar = maximum(abs.(F.values)) # HACK: okay since we are ignoring the hard case
  22.045702676755557
julia> y = F.vectors[:, end] # And a worse hack
julia> y1, y2 = y[1:10], y[11:20]
julia> pstar = sign(g'*y2)*Δ*y1./sqrt(y1'*B*y1)
10-element Vector{ComplexF64}:
   -0.0696182291622581 + 0.0im
 -0.029856150657183066 + 0.0im
  0.021457658057046993 + 0.0im
 -0.015672083161465084 + 0.0im
   0.03743652574264288 + 0.0im
  -0.04079621450785599 + 0.0im
 -0.017286212026279373 + 0.0im
   0.01829971992538563 + 0.0im
  -0.00794105054155718 + 0.0im
 -0.056551058280238306 + 0.0im
julia> sqrt(pstar'*B*pstar) 
 0.1 + 0.0im # Hence our solution is on the boundary
\end{lstlisting}
\end{solution}
\end{document} 